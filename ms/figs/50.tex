% latex table generated in R 4.0.2 by xtable 1.8-4 package
% Tue Dec  8 16:13:58 2020
\begin{table}[ht]
\centering
\begin{tabular}{r|cccc}
  \hline
$\sigma_{\epsilon}^2$ & Normal & Bimodal 1 & Bimodal 2 & Trimodal \\ 
  \hline
2 & (1.02,1.18) & (1.08,1.01) & (1.03,1.02) & (1.18,1.05) \\ 
  1 & (1.05,1.17) & (1.15,1.01) & (1.07,1.03) & (1.24,1.04) \\ 
  0.5 & (1.13,1.11) & (1.26,1.01) & (1.16,1.03) & (1.30,1.01) \\ 
  0.25 & (1.32,1.05) & (1.50,1.00) & (1.37,1.01) & (1.46,1.00) \\ 
  0.125 & (1.70,1.02) & (1.92,1.00) & (1.76,1.01) & (1.77,1.00) \\ 
   \hline
\end{tabular}
\caption{Each entry is $\left(\frac{MISE(0)}{MISE(h_Y)},\frac{MISE(h_X)}{MISE(h_Y)}\right)$ for $n=50$. These ratios are always greater than $1$ because $h_Y$ is the minimizer of the $\MISE$. As expected, $\MISE(0)$ performs well when $\sigma_{\epsilon}^2$ (the error variance) is large but poorly when $\sigma_{\epsilon}^2$ is small. $\MISE(h_X)$ performs well when $\sigma_{\epsilon}^2$ is small but poorly when $\sigma_{\epsilon}^2$ is large.} 
\label{tab:50}
\end{table}
