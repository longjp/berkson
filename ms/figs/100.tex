% latex table generated in R 4.0.2 by xtable 1.8-4 package
% Tue Dec  8 16:13:59 2020
\begin{table}[ht]
\centering
\begin{tabular}{r|cccc}
  \hline
$\sigma_{\epsilon}^2$ & Normal & Bimodal 1 & Bimodal 2 & Trimodal \\ 
  \hline
2 & (1.01,1.24) & (1.04,1.03) & (1.02,1.04) & (1.09,1.02) \\ 
  1 & (1.03,1.24) & (1.08,1.03) & (1.04,1.06) & (1.12,1.01) \\ 
  0.5 & (1.07,1.18) & (1.15,1.03) & (1.09,1.06) & (1.16,1.00) \\ 
  0.25 & (1.19,1.09) & (1.31,1.02) & (1.24,1.03) & (1.27,1.00) \\ 
  0.125 & (1.46,1.04) & (1.62,1.01) & (1.53,1.01) & (1.50,1.00) \\ 
   \hline
\end{tabular}
\caption{The entries here are the same as Table \ref{tab:50} but for $n=100$. This larger $n$ generally improves performance for $\MISE(0)$ and worsens the performance of $\MISE(h_X)$ (relative to $\MISE(h_Y)$). This is predicted by our asymptotic theory, since as $n \rightarrow \infty$, $\frac{MISE(0)}{MISE(h_Y)} \rightarrow 1$ while $\frac{MISE(h_X)}{MISE(h_Y)} \rightarrow \infty$. However at $n=100$, using $h_X$ still generally outperforms no smoothing.} 
\label{tab:100}
\end{table}
